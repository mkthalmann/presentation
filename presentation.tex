%!TEX program = xelatex
\documentclass[10pt,ignorenonframetext,,aspectratio=149]{beamer}
\usetheme{metropolis}
\usefonttheme{serif} % use mainfont rather than sansfont for slide text
\setbeamertemplate{caption}[numbered]
\setbeamertemplate{caption label separator}{: }
\setbeamercolor{caption name}{fg=normal text.fg}
\usepackage{amssymb,amsmath}
\usepackage{ifxetex,ifluatex}
\usepackage{fixltx2e} % provides \textsubscript
\ifnum 0\ifxetex 1\fi\ifluatex 1\fi=0 % if pdftex
  \usepackage[T1]{fontenc}
  \usepackage[utf8]{inputenc}
  \else % if luatex or xelatex
  \ifxetex
    \usepackage{unicode-math}
  \else
    \usepackage{fontspec}
  \fi
  \defaultfontfeatures{Ligatures=TeX,Scale=MatchLowercase}
  \newcommand{\euro}{€}
    \setmainfont[Scale=MatchLowercase,]{Roboto Condensed}
      \setsansfont[Scale=MatchLowercase,]{Roboto Condensed}
      \setmonofont[Mapping=tex-ansi,Scale=MatchLowercase]{Roboto Mono}
      \setmathfont{Asana Math}
  \fi
% use upquote if available, for straight quotes in verbatim environments
\usepackage{upquote}
\usepackage{microtype}
\UseMicrotypeSet[protrusion]{basicmath}

\usepackage{polyglossia}
\setmainlanguage{english}
\setotherlanguage{ngerman}

\usepackage[
  backend=biber,
  bibstyle=biblatex-sp-unified,
  citestyle=sp-authoryear-comp,
  maxcitenames=3,
  maxbibnames=99,
  url=false
]{biblatex}
\addbibresource{../bibliographyuni.bib}
\usepackage{color}
\usepackage{fancyvrb}
\newcommand{\VerbBar}{|}
\newcommand{\VERB}{\Verb[commandchars=\\\{\}]}
\DefineVerbatimEnvironment{Highlighting}{Verbatim}{commandchars=\\\{\}}
% Add ',fontsize=\small' for more characters per line
\usepackage{framed}
\definecolor{shadecolor}{RGB}{35,38,41}
\newenvironment{Shaded}{\begin{snugshade}}{\end{snugshade}}
\newcommand{\AlertTok}[1]{\textcolor[rgb]{0.58,0.85,0.30}{\textbf{\colorbox[rgb]{0.30,0.12,0.14}{#1}}}}
\newcommand{\AnnotationTok}[1]{\textcolor[rgb]{0.25,0.50,0.35}{#1}}
\newcommand{\AttributeTok}[1]{\textcolor[rgb]{0.16,0.50,0.73}{#1}}
\newcommand{\BaseNTok}[1]{\textcolor[rgb]{0.96,0.45,0.00}{#1}}
\newcommand{\BuiltInTok}[1]{\textcolor[rgb]{0.50,0.55,0.55}{#1}}
\newcommand{\CharTok}[1]{\textcolor[rgb]{0.24,0.68,0.91}{#1}}
\newcommand{\CommentTok}[1]{\textcolor[rgb]{0.48,0.49,0.49}{#1}}
\newcommand{\CommentVarTok}[1]{\textcolor[rgb]{0.50,0.55,0.55}{#1}}
\newcommand{\ConstantTok}[1]{\textcolor[rgb]{0.15,0.68,0.68}{\textbf{#1}}}
\newcommand{\ControlFlowTok}[1]{\textcolor[rgb]{0.99,0.74,0.29}{\textbf{#1}}}
\newcommand{\DataTypeTok}[1]{\textcolor[rgb]{0.16,0.50,0.73}{#1}}
\newcommand{\DecValTok}[1]{\textcolor[rgb]{0.96,0.45,0.00}{#1}}
\newcommand{\DocumentationTok}[1]{\textcolor[rgb]{0.64,0.20,0.25}{#1}}
\newcommand{\ErrorTok}[1]{\textcolor[rgb]{0.85,0.27,0.33}{\underline{#1}}}
\newcommand{\ExtensionTok}[1]{\textcolor[rgb]{0.00,0.60,1.00}{\textbf{#1}}}
\newcommand{\FloatTok}[1]{\textcolor[rgb]{0.96,0.45,0.00}{#1}}
\newcommand{\FunctionTok}[1]{\textcolor[rgb]{0.56,0.27,0.68}{#1}}
\newcommand{\ImportTok}[1]{\textcolor[rgb]{0.15,0.68,0.38}{#1}}
\newcommand{\InformationTok}[1]{\textcolor[rgb]{0.77,0.36,0.00}{#1}}
\newcommand{\KeywordTok}[1]{\textcolor[rgb]{0.81,0.81,0.76}{\textbf{#1}}}
\newcommand{\NormalTok}[1]{\textcolor[rgb]{0.81,0.81,0.76}{#1}}
\newcommand{\OperatorTok}[1]{\textcolor[rgb]{0.81,0.81,0.76}{#1}}
\newcommand{\OtherTok}[1]{\textcolor[rgb]{0.15,0.68,0.38}{#1}}
\newcommand{\PreprocessorTok}[1]{\textcolor[rgb]{0.15,0.68,0.38}{#1}}
\newcommand{\RegionMarkerTok}[1]{\textcolor[rgb]{0.16,0.50,0.73}{\colorbox[rgb]{0.08,0.19,0.26}{#1}}}
\newcommand{\SpecialCharTok}[1]{\textcolor[rgb]{0.24,0.68,0.91}{#1}}
\newcommand{\SpecialStringTok}[1]{\textcolor[rgb]{0.85,0.27,0.33}{#1}}
\newcommand{\StringTok}[1]{\textcolor[rgb]{0.96,0.31,0.31}{#1}}
\newcommand{\VariableTok}[1]{\textcolor[rgb]{0.15,0.68,0.68}{#1}}
\newcommand{\VerbatimStringTok}[1]{\textcolor[rgb]{0.85,0.27,0.33}{#1}}
\newcommand{\WarningTok}[1]{\textcolor[rgb]{0.85,0.27,0.33}{#1}}

% Comment these out if you don't want a slide with just the
% part/section/subsection/subsubsection title:
% \AtBeginPart{
%   \let\insertpartnumber\relax
%   \let\partname\relax
%   \frame{\partpage}
% }
% \AtBeginSection{
%   \let\insertsectionnumber\relax
%   \let\sectionname\relax
%   \frame{\sectionpage}
% }
% \AtBeginSubsection{
%   \let\insertsubsectionnumber\relax
%   \let\subsectionname\relax
%   \frame{\subsectionpage}
% }

\setlength{\emergencystretch}{3em}  % prevent overfull lines
\providecommand{\tightlist}{%
  \setlength{\itemsep}{0pt}\setlength{\parskip}{0pt}}
\setcounter{secnumdepth}{5}
\title{An Example R Markdown Document}
\subtitle{(A Subtitle Would Go Here if This Were a Class)}
\author{Maik Thalmann}
\date{Where, 12 April, 2020}

%% Here's everything I added.
%%--------------------------

\usepackage{ltablex}
\usepackage{booktabs}
\renewcommand{\arraystretch}{1.03}
\newcolumntype{L}{>{\scriptsize}l}

% colors
\usepackage{tikz}
\definecolor{mygreen}{RGB}{159, 214, 223} % 0, 19, 108
\definecolor{myred}{RGB}{252, 127, 178} %        96, 106, 179;
\definecolor{background-color}{RGB}{40, 40, 40} % #48525c
\definecolor{shadecolor}{RGB}{63, 63, 64}
\newcommand*{\mygreen}[1]{\textcolor{mygreen}{\textbf{#1}}}
\newcommand*{\myred}[1]{\textcolor{myred}{\textbf{#1}}}

% set up tikz
\usepackage{tikz-qtree}
\def\mysize{\footnotesize}
\usetikzlibrary{decorations.text,backgrounds,arrows.meta,decorations.pathreplacing,calc, fit, shapes,positioning,intersections,decorations.markings}
\tikzset{
  font=\color{white},
  every tree node/.style={align=center,anchor=base},
  edge from parent/.append style={ultra thin, white}}
\tikzset{semtree/.style={
      level distance=2em,
      every tree node/.style={font=\bfseries}}
}

\usepackage{rotating}
\usepackage{hyperref}
\hypersetup{
  colorlinks  = true,
  urlcolor    = mygreen,
  linkcolor   = mygreen,
  citecolor   = mygreen,
  pdfauthor   = {Maik Thalmann},
  pdfproducer = {XeLaTeX},
  pdfsubject  = {Linguistics; University of Göttingen},
}

% packages necessary for the semantics
\usepackage{latexsym}
\usepackage{mathtools}
\usepackage[cal=esstix]{mathalfa}
\usepackage{upgreek}
\usepackage{wasysym}
\usepackage{stmaryrd}
\usepackage{soul}
\newcommand{\sem}[1]{\mbox{$\llbracket$\sffamily\textbf{#1}$\rrbracket$}}

\usepackage{relsize}
\usepackage[framemethod=tikz]{mdframed}
% custom rule environment
\newmdenv[
  linecolor=mygreen,
  backgroundcolor=background-color,
  fontcolor=white,
  linewidth=2pt,
  topline=false,
  bottomline=false,
  rightmargin=0,
  leftmargin=0,
  innerrightmargin=5pt,
  innerleftmargin=5pt
]{rulebox}
% custom quotation environment
\newmdenv[
  linecolor=mygreen,
  backgroundcolor=background-color!80,
  fontcolor=white,
  linewidth=4pt,
  font=\smaller,
  topline=false,
  bottomline=false,
  rightline=false,
  innerrightmargin=5pt,
  innerleftmargin=5pt
]{quotebox}

% Optional institute tags and titlegraphic.
% Do feel free to change the titlegraphic if you don't want it as a Markdown field.
%----------------------------------------------------------------------------------
\institute{Georg-August-University Göttingen}


% Some beamer adjustments
%----------------------------------------
\setbeamertemplate{title page}[empty]
\setbeamertemplate{frametitle}{
  \strut\insertframetitle\strut
}
\addtobeamertemplate{frametitle}{\vskip1ex}{}
\setbeamertemplate{section page}
{
  \vskip-25ex
  \begin{centering}
    \usebeamerfont{section title}\insertsection\par
    \usebeamertemplate*{progress bar in section page}
  \end{centering}
}
\setbeamertemplate{frametitle continuation}[from second][cont'd]
\setbeamertemplate{navigation symbols}{}
\setbeamertemplate{bibliography item}{}
\setbeamertemplate{itemize item}{\color{white}$\bullet$}
\setbeamertemplate{itemize subitem}{\color{white}\scriptsize{$\bullet$}}
\setbeamertemplate{itemize/enumerate body end}{\vspace{.6\baselineskip}}
\setbeamercovered{transparent}
\setbeamercolor{background canvas}{bg=background-color}
\setbeamercolor{frametitle}{fg=mygreen, bg=background-color}
\setbeamercolor{title}{fg=white}
\setbeamercolor{normal text}{fg=white}
\setbeamercolor{alerted text}{fg=myred}
\setbeamercolor{local structure}{fg=white}
\setbeamercolor{section in toc}{fg=mygreen}
\setbeamercolor{structure}{fg=white}
\setbeamercolor{local structure}{parent=structure}
\setbeamercolor{item projected}{parent=item,use=item,fg=white}
\setbeamercolor{enumerate item}{parent=item}
\setbeamercolor{footline}{fg=mygreen}
\setbeamercolor{block title}{fg=myred,bg=white}
\setbeamercolor{bibliography entry author}{fg=white}
\setbeamercolor{bibliography entry location}{fg=white}
\setbeamercolor{bibliography entry title}{fg=white}
\setbeamercolor{bibliography entry note}{fg=white}
\setbeamercolor{bibliography entry item}{fg=white}
\setbeamerfont{title}{family=\fontspec{Canela}, size=\huge}
\setbeamerfont{section title}{family=\fontspec{Canela}, size=\huge}
\setbeamerfont{subtitle}{family=\fontspec{Canela}}
\setbeamerfont{frametitle}{family=\fontspec{Canela}, size=\LARGE}

% Sections and subsections should not get their own damn slide.
%--------------------------------------------------------------
\AtBeginSubsection{}
\AtBeginSubsubsection{}

% Suppress some of Markdown's weird default vertical spacing.
%------------------------------------------------------------
\setlength{\emergencystretch}{0em}  % prevent overfull lines
\setlength{\parskip}{0pt}

% two-tone footlines
%--------------------------------------------------
\defbeamertemplate*{footline}{my footline}{%
  \ifnum\insertpagenumber=1
    \hbox{%
      \begin{beamercolorbox}[wd=\paperwidth,ht=.8ex,dp=1ex,center]{}%
        % empty environment to raise height
      \end{beamercolorbox}%
    }%
    \vskip0pt%
  \else%
    \scriptsize{%
      \hspace*{0.1cm}\url{https://mkthalmann.github.io/home/}\hfill%
      \vspace*{1pt}%
      \insertframenumber/\inserttotalframenumber \hspace*{0.1cm}%
      \newline%
      \color{mygreen}{\rule{\paperwidth}{0.3mm}}\newline%
      \color{myred}{\rule{\paperwidth}{.4mm}}%
    }%
  \fi%
}

\usepackage{gb4e}
\usepackage{etoolbox}

\AtBeginEnvironment{Shaded}{\smaller}
\AtBeginEnvironment{verbatim}{\smaller}
\definecolor{basiccolor}{rgb}{0.81,0.81,0.76}
\surroundwithmdframed[
  backgroundcolor=shadecolor,
  fontcolor=basiccolor,
  leftmargin=0,
  rightmargin=0,
  innerleftmargin=2pt,
  innerrightmargin=2pt,
  hidealllines
  ]{verbatim}


\usepackage[font=small,skip=0pt]{caption}


\begin{document}
\frame{\titlepage}

\begin{frame}{Outline}
  \tableofcontents
\end{frame}

\hypertarget{pop-songs-and-political-science}{%
\section{Pop Songs and Political
Science}\label{pop-songs-and-political-science}}

\begin{frame}{Sheena Easton and Game Theory}
\protect\hypertarget{sheena-easton-and-game-theory}{}
Sheena Easton describes the following scenario for her baby:

\begin{enumerate}
\tightlist
\item
  Takes the morning train
\item
  Works from nine 'til five
\item
  Takes another train home again
\item
  Finds Sheena Easton waiting for him
\end{enumerate}
\end{frame}

\begin{frame}[fragile,allowframebreaks]{R Stuff}
\protect\hypertarget{r-stuff}{}
\begin{verbatim}
# A tibble: 10 x 10
   carat cut       color clarity depth table price     x     y     z
   <dbl> <ord>     <ord> <ord>   <dbl> <dbl> <int> <dbl> <dbl> <dbl>
 1 0.23  Ideal     E     SI2      61.5    55   326  3.95  3.98  2.43
 2 0.21  Premium   E     SI1      59.8    61   326  3.89  3.84  2.31
 3 0.23  Good      E     VS1      56.9    65   327  4.05  4.07  2.31
 4 0.290 Premium   I     VS2      62.4    58   334  4.2   4.23  2.63
 5 0.31  Good      J     SI2      63.3    58   335  4.34  4.35  2.75
 6 0.24  Very Good J     VVS2     62.8    57   336  3.94  3.96  2.48
 7 0.24  Very Good I     VVS1     62.3    57   336  3.95  3.98  2.47
 8 0.26  Very Good H     SI1      61.9    55   337  4.07  4.11  2.53
 9 0.22  Fair      E     VS2      65.1    61   337  3.87  3.78  2.49
10 0.23  Very Good H     VS1      59.4    61   338  4     4.05  2.39
\end{verbatim}

Some text to compare font sizes on this slide.

\begin{Shaded}
\begin{Highlighting}[]
\KeywordTok{library}\NormalTok{(psych)}
\NormalTok{desc \textless{}{-}}\StringTok{ }\KeywordTok{as.data.frame}\NormalTok{(}\KeywordTok{describeBy}\NormalTok{(d}\OperatorTok{$}\NormalTok{price, d}\OperatorTok{$}\NormalTok{color, }\DataTypeTok{mat =}\NormalTok{ T, }\DataTypeTok{digits =} \DecValTok{2}\NormalTok{))}
\KeywordTok{kable}\NormalTok{(desc, }\DataTypeTok{booktabs =}\NormalTok{ T) }\OperatorTok{\%\textgreater{}\%}
\StringTok{  }\KeywordTok{kable\_styling}\NormalTok{(}\DataTypeTok{latex\_options =} \StringTok{"scale\_down"}\NormalTok{)}
\end{Highlighting}
\end{Shaded}

\begin{table}[H]
\centering
\resizebox{\linewidth}{!}{
\begin{tabular}{lllrrrrrrrrrrrrr}
\toprule
  & item & group1 & vars & n & mean & sd & median & trimmed & mad & min & max & range & skew & kurtosis & se\\
\midrule
X11 & 1 & D & 1 & 6775 & 3169.95 & 3356.59 & 1838.0 & 2457.57 & 1657.55 & 357 & 18693 & 18336 & 2.10 & 4.67 & 40.78\\
X12 & 2 & E & 1 & 9797 & 3076.75 & 3344.16 & 1739.0 & 2349.98 & 1537.46 & 326 & 18731 & 18405 & 2.17 & 4.89 & 33.79\\
X13 & 3 & F & 1 & 9542 & 3724.89 & 3784.99 & 2343.5 & 2974.69 & 2274.31 & 342 & 18791 & 18449 & 1.75 & 2.82 & 38.75\\
X14 & 4 & G & 1 & 11292 & 3999.14 & 4051.10 & 2242.0 & 3245.61 & 2277.27 & 354 & 18818 & 18464 & 1.50 & 1.72 & 38.12\\
X15 & 5 & H & 1 & 8304 & 4486.67 & 4215.94 & 3460.0 & 3755.13 & 3683.52 & 337 & 18803 & 18466 & 1.38 & 1.45 & 46.26\\
\addlinespace
X16 & 6 & I & 1 & 5422 & 5091.87 & 4722.39 & 3730.0 & 4332.86 & 4067.51 & 334 & 18823 & 18489 & 1.16 & 0.42 & 64.13\\
X17 & 7 & J & 1 & 2808 & 5323.82 & 4438.19 & 4234.0 & 4721.87 & 4088.27 & 335 & 18710 & 18375 & 1.03 & 0.28 & 83.75\\
\bottomrule
\end{tabular}}
\end{table}
\end{frame}

\begin{frame}{Plot}
\protect\hypertarget{plot}{}
\begin{center}\includegraphics[width=0.9\linewidth]{figs/unnamed-chunk-3-1} \end{center}
\end{frame}

\begin{frame}{Rick Astley's Re-election Platform}
\protect\hypertarget{rick-astleys-re-election-platform}{}
Rick Astley's campaign promises:

\begin{itemize}
\tightlist
\item
  Never gonna give you up.
\item
  Never gonna let you down.
\item
  Never gonna run around and desert you.
\item
  Never gonna make you cry.
\item
  Never gonna say goodbye.
\item
  Never gonna tell a lie and hurt you.
\end{itemize}

Are these promises (if credible) sufficient to secure re-election?
\end{frame}

\begin{frame}{Rick Astley and Median Voter Theorem}
\protect\hypertarget{rick-astley-and-median-voter-theorem}{}
Whereas these pledges conform to the preferences of the \textbf{median
voter}, we expect Congressman Astley to secure re-election.
\end{frame}

\begin{frame}{Caribbean Queen and Operation Urgent Fury}
\protect\hypertarget{caribbean-queen-and-operation-urgent-fury}{}
Billy Ocean released ``Caribbean Queen'' in 1984.

\begin{itemize}
\tightlist
\item
  Emphasized sharing the same dream
\item
  Hearts beating as one
\end{itemize}

``Caribbean Queen'' is about the poor execution of Operation Urgent
Fury.

\begin{itemize}
\tightlist
\item
  Echoed JCS chairman David Jones' frustrations with military
  establishment.
\end{itemize}

Billy Ocean is advocating for what became the Goldwater-Nichols Act.

\begin{itemize}
\tightlist
\item
  Wanted to take advantage of \textbf{economies of scale}, resolve
  \textbf{coordination problems} in U.S. military.
\end{itemize}
\end{frame}

\begin{frame}{The Good Day Hypothesis}
\protect\hypertarget{the-good-day-hypothesis}{}
We know the following about Ice Cube's day.

\begin{enumerate}
\tightlist
\item
  The Lakers beat the Supersonics.
\item
  No helicopter looked for a murder.
\item
  Consumed \alert{Fatburger} at 2 a.m.
\item
  Goodyear blimp: ``Ice Cube's a pimp.''
  \textcite{heimkratzer1998semantics}
\item
  \autocites{posner1980semantics,hintikka1969semantics,gries2013statistics,grice1989studies,groenendijkstokhof1984questions}
\end{enumerate}
\end{frame}

\begin{frame}{The Good Day Hypothesis}
\protect\hypertarget{the-good-day-hypothesis-1}{}
\begin{tikzpicture}[baseline=(top.base), scale=.8]
    \Tree [.\node(top){DP}; [.D der ] [.NP [.AP [.A große ] ] [.N$''''$ [.AP^2 [.A^2 verschüchterte ] ] [.N$'''$ [.AP^3 [.A^3 fliegende ] ] [.N$''$ [.N$'$ [.N Wolf ] ] [.PP [.P aus ] [.NP^2 [.N^2 Twilight ] ] ] ] ] ] ] ]
\end{tikzpicture}
\end{frame}

\begin{frame}{The Good Day Hypothesis}
\protect\hypertarget{the-good-day-hypothesis-2}{}
\begin{quotebox}
    Colorless green ideas sleep furiously\hfill (Noam Chomsky)
\end{quotebox}

\begin{rulebox}
    \textbf{\color{mygreen} Functional Application (FA)}\hfill {\footnotesize H\&K:49}\\Wenn $\alpha$ ein verzweigender Knoten ist, \{$\beta$, $\gamma$\} die Menge von $\alpha$'s Töchtern ist und \sem{$\symbf{\beta}$} eine Funktion ist, dessen Domäne \sem{$\symbf{\gamma}$} enthält, dann \sem{$\symbf{\alpha}$} = \sem{$\symbf{\beta}$}(\sem{$\symbf{\gamma}$}).
\end{rulebox}
\end{frame}

\begin{frame}{Semantics}
\protect\hypertarget{semantics}{}
\begin{exe}
  \ex Lexikoneinträge
  
  \sem{not} = $\lambda p \in D_t$ . $p=0$\\
    \sem{Carla} = Carla\\
    \sem{invite} = $\lambda x \in D_e$ . [$\lambda y \in D_e$ . $y$ lädt $x$ ein]\\
    \sem{a} = $\lambda f \in D_{\langle e, t\rangle}$ . $[\lambda g \in D_{\langle e, t\rangle}$ . es gibt ein $x$, sodass $f(x)=1$ und $g(x)=1]$\\
    \sem{politician} = $\lambda x \in D_e$ . $x$ ist ein Politiker
\end{exe}
\end{frame}

\begin{frame}{The Good Day Hypothesis}
\protect\hypertarget{the-good-day-hypothesis-3}{}
This leads to two different hypotheses:

\begin{itemize}
\tightlist
\item
  \(H_0\): Ice Cube's day is statistically indistinguishable from a
  typical day.
\item
  \(H_1\): Ice Cube is having a good (i.e.~greater than average) day.
\end{itemize}

These hypotheses are tested using archival data of Ice Cube's life.
\end{frame}

\hypertarget{all-code}{%
\section*{All Code}\label{all-code}}

\hypertarget{all-the-code-i-used}{%
\subsection*{All The Code I Used}\label{all-the-code-i-used}}

\begin{frame}[fragile,allowframebreaks]{All The Code I Used}
\begin{Shaded}
\begin{Highlighting}[]
\NormalTok{d \textless{}{-}}\StringTok{ }\NormalTok{diamonds}
\KeywordTok{head}\NormalTok{(diamonds, }\DecValTok{10}\NormalTok{)}
\KeywordTok{library}\NormalTok{(psych)}
\NormalTok{desc \textless{}{-}}\StringTok{ }\KeywordTok{as.data.frame}\NormalTok{(}\KeywordTok{describeBy}\NormalTok{(d}\OperatorTok{$}\NormalTok{price, d}\OperatorTok{$}\NormalTok{color, }\DataTypeTok{mat =}\NormalTok{ T, }\DataTypeTok{digits =} \DecValTok{2}\NormalTok{))}
\KeywordTok{kable}\NormalTok{(desc, }\DataTypeTok{booktabs =}\NormalTok{ T) }\OperatorTok{\%\textgreater{}\%}
\StringTok{  }\KeywordTok{kable\_styling}\NormalTok{(}\DataTypeTok{latex\_options =} \StringTok{"scale\_down"}\NormalTok{)}
\KeywordTok{ggplot}\NormalTok{(d, }\KeywordTok{aes}\NormalTok{(}\DataTypeTok{x =}\NormalTok{ carat, }\DataTypeTok{y =}\NormalTok{ price)) }\OperatorTok{+}
\StringTok{  }\KeywordTok{stat\_summary}\NormalTok{(}\DataTypeTok{fun.y =}\NormalTok{ mean, }\DataTypeTok{geom =} \StringTok{"line"}\NormalTok{, }\DataTypeTok{color =}\NormalTok{ mygreen) }\OperatorTok{+}
\StringTok{  }\KeywordTok{theme\_maik}\NormalTok{()}
\end{Highlighting}
\end{Shaded}
\end{frame}

\begin{frame}[allowframebreaks]{}
  \printbibliography[heading=none]
\end{frame}


\end{document}